\documentclass[a4paper,11pt]{article}
\usepackage[utf8]{inputenc}
\usepackage{subcaption}
\usepackage[siunitx]{circuitikz}
\usepackage{longtable}
\usepackage{titlesec}
\usepackage{graphicx}
\usepackage{amsmath}
\usepackage{float}
\usepackage[margin=2cm]{geometry}
\usepackage[justification=centering]{caption}
\DeclareGraphicsExtensions{.png}

\titleformat*{\section}{\sffamily\LARGE\bfseries}
\titleformat*{\subsection}{\sffamily\Large\bfseries}
\titleformat*{\subsubsection}{\sffamily\large\bfseries}

\begin{document}
\section*{Bauelementegleichungen}
\subsection*{Diode}
\subsection*{MOSFET}
\section*{Prozessparameter}
\section*{Parasitäre Kapazitäten}
\section*{Verzögerungszeiten}
\subsection*{Laden / Entladen einer Kapazität}
\[
	i(t) = -C_L ~ \frac{\mathrm du_c(t)}{\mathrm dt}
\]

Problem: Finde $t$, stelle also um:
\[
	\mathrm dt = - \frac{C_L}{i(t)} ~ \mathrm du_c
\]

Wobei $i(t)$ durch die Gleichung des entsprechendes MOSFETs bestimmt wird, d.h. nur von $u_c$ abhängt. Die Gesamtzeit ergibt sich dann durch Integration:
\[
	t_{charge} = - \int_{U_{start}}^{U_{end}} \frac{C_L}{I_{DS}(u_c)} ~ \mathrm du_c
\]

In der Regel muss dabei eine Fallunterscheidung für die verschiedenen Betriebsbereiche des MOSFETs getroffen werden mit dazugehörigem $I_{DS}$.

Wichtig: Als Endwert sollte nicht die vollständige Umladung (also $U_{DD}$ oder $0V$) verwendet werden, da dieser Wert mathematisch nie erreicht wird.

\subsubsection*{Kapazität über NMOS entladen}

\begin{figure}[H]
\centering
\begin{subfigure}{.35\textwidth}
	\centering
	\begin{circuitikz}[european, scale=0.7]
		\draw
			(0,0) node[nigfetebulk](nmos1){}
			(nmos1.S) to node[ground]{} (0, -1)
			(nmos1.D) to ++(0, 0.5) to [short,i<=$i(t)$] ++(1.5, 0) to [C,l^=$C_L$] (1.5, -1) to node[ground]{} (1.5, -1)
			(nmos1.G) to [short, -o] ++(-1.5, 0) node[label={[font=\footnotesize]above:$U_G(t) = U_{DD}$}]{}
		;
	\end{circuitikz}
\end{subfigure}
\begin{subfigure}{.49\textwidth}
	\begin{tabular}{r l}
		\textbf{Startspannung} & $U_{start} = U_{DD}$ \\
		\textbf{Zielspannung} & $U_{end} = 0.1 ~ U_{DD}$ \\
		\textbf{Threshold-Spannung} & $U_{Tn} = 0.2 ~ U_{DD}$ \\
		\textbf{Kanallängenmodulation} & vernachlässigt, $\lambda = 0$
	\end{tabular}
\end{subfigure}
\end{figure}

Abfallzeit:
\[
	\boxed{ \quad t_{HL} \approx 4 ~ \frac{C_L}{\beta_n ~ U_{DD}} \quad }
\]

\subsubsection*{Kapazität über PMOS laden}

\begin{figure}[H]
\centering
\begin{subfigure}{.35\textwidth}
	\centering
	\begin{circuitikz}[european, scale=0.7]
		\draw
			(0,0) node[pigfetebulk, rotate=90](pmos1){}
			(pmos1.G) to [short] ++(0, -1) node[ground,label={[font=\footnotesize]left:$U_G(t) = 0V$}]{}
			(pmos1.S) to [short, -o] ++(-1, 0) node[label={[font=\footnotesize]above:$U_{DD}$}]{}
			(pmos1.D) to [short,i<=$i(t)$] ++(1, 0) to [C,l^=$C_L$] ++(0,-2.4) node[ground]{}
		;
	\end{circuitikz}
	\caption*{Beachte: $i(t) < 0$}
\end{subfigure}
\begin{subfigure}{.49\textwidth}
	\begin{tabular}{r l}
		\textbf{Startspannung} & $U_{start} = 0V$ \\
		\textbf{Zielspannung} & $U_{end} = 0.9 ~ U_{DD}$ \\
		\textbf{Threshold-Spannung} & $U_{Tp} = -0.2 ~ U_{DD}$ \\
		\textbf{Kanallängenmodulation} & vernachlässigt, $\lambda = 0$
	\end{tabular}
\end{subfigure}
\end{figure}

Anstiegszeit:
\[
	\boxed{ \quad t_{LH} \approx 4 ~ \frac{C_L}{\beta_p ~ U_{DD}} \quad }
\]

\end{document}